\section{Introduction}\label{sec:introduction}
% Our group and what are we doing
This report aims at providing a brief explanation of the Information Retrieval system built
as a team project during the Search Engine course 22/23 of the master degree in Computer Engineering and Data Science at University of Padua, Italy. As a group in this subject, we are participating in the 2023 CLEF LongEval: Longitudinal Evaluation of Model Performance. This is the annual evaluation campaign that focuses on the longitudinal evaluation of model performance in information
retrieval and natural language processing.\\
% The LongEval corpus
The LongEval collection\cite{traindata} relies on a large set of data provided by Qwant (a commercial privacy-focused
search engine that was launched in France in 2013). %TODO: link to Qwant
Their idea regarding the dataset was to reflect changes of the Web across time, providing evolving document and query
sets.
The data was collected in June 2022.
The dataset consists of 672 training \textbf{queries}, 9656 corresponding evaluation assignments, and 98 held-out queries.
The collection's \textbf{documents} were chosen based on queries using the Qwant click model, in addition to random
selection of documents from the Qwant index.
The training queries are categorized into twenty \textbf{topics}, including car-related, antivirus-related,
employment-related, energy-related, recipe-related, rental car-related, video-related, war-related, gateau-related,
police-related, tax-related, solar panel-related, water-related, elderberry-related, curtain-related, land-related,
retirement-related, coronavirus-related, beauty-related, and miscellaneous queries.
In addition to the original French versions, the collection also includes English translations of the webpages and
queries using the CUBBITT system.\\
% Our approach
Our approach considers both the English and French versions of the documents.
We generate character N-grams to identify common word structures (as prefixes or suffixes) repeated over documents.
We also use query expansion with synonyms (in English) and the Natural Language Processing (NLP) technique of Named
Entity Recognition (NER) to further refine our system.
Our system was developed in Java, mainly using the library Lucene.\\
% Organization
The paper is organized as follows:
Section~\ref{sec:methodology} briefly describes our approach;
Section~\ref{sec:architecture} describes our code in detail;
Section~\ref{sec:setup} explains our experimental setup;
Section~\ref{sec:results} discusses our main findings; finally,
Section~\ref{sec:conclusion} draws some conclusions and outlooks for future work.