\section{Introduction}\label{sec:introduction}
% Our group and what are we doing
This report aims at providing a brief explanation of the Information Retrieval system built as a team project during the
Search Engine course 22/23 of the master's degree in Computer Engineering and Data Science at the University of Padua,
Italy.
As a group in this subject, we are participating in the 2023 CLEF LongEval: Longitudinal Evaluation of Model
Performance~\cite{LongEval}.
This annual evaluation campaign focuses on the longitudinal evaluation of model performance in information retrieval and
natural language processing.\\

% The LongEval corpus
The LongEval collection~\cite{traindata} relies on a large set of data provided by Qwant (a commercial privacy-focused
search engine that was launched in France in 2013).
Their idea regarding the dataset (collected in June 2022) was to reflect changes of the Web across time, providing
evolving document and query sets.
The training collection consists of 672 \textbf{queries}, 98 held-out queries, and 9656 evaluation assignments.
The \textbf{documents} were chosen based on queries using the Qwant click model, in addition to random selection from
the Qwant index.
The training queries are categorized into twenty \textbf{topics}, such as: car-related, antivirus-related,
employment-related, energy-related, recipe-related, etc.
In addition to the original French version, the collection also includes English translations of the documents and
queries using the CUBBITT~\cite{CUBBITT} system.\\

% Organization
The paper is organized as follows:
Section~\ref{sec:methodology} briefly describes our approach;
Section~\ref{sec:architecture} describes our code in detail;
Section~\ref{sec:setup} explains our experimental setup;
Section~\ref{sec:results} discusses our main findings; finally,
Section~\ref{sec:conclusion} draws some conclusions and outlooks for future work.