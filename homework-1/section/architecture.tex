\section{System Architecture}\label{sec:architecture}

In this section, we address the technical aspects of how our system was developed.
Some parts of the code we are going to present are inspired in the repositories developed by Professor Nicola Ferro and
presented to us during the Search Engine course.
We will follow the structure (in packages) of the repository~\cite{jihuming}.

\subsection{Parsing}\label{subsec:parsing}

To generate an index based on the provided documents, first, we have to parse them;
this is, get their text into Java data structures.
All this parser package has been developed following the instructions provided by the organizers.\\

We decided to use the \textbf{JSON} version of the documents because they can easily be manipulated and queried using a
variety of tools and libraries.
In our case, because of the large number of documents we are working with, we had to create a \textbf{streaming} parser,
that allocates into the main memory only one document at a time.
Thus, our parser is based on the Java library \texttt{Gson} including streaming funtionalities.\\

The whole parser is made up of the following classes:
\begin{itemize}
    \item \texttt{DocumentParser}: an abstract class representing a streaming parser.
    Implements \texttt{Iterator} and \text{Iterable}.
    \item \texttt{JsonDocument}: a Java POJO for the deserialization of JSON documents.
    \item \texttt{ParsedDocument}: represents a document already parsed.
    Note that this class only contains an identifier and a body.
    \item \texttt{LongEvalParser}: real parser implementing the class \texttt{DocumentParser}.
    Here is where the streaming logic is implemented.
    Objects of this class can be used as iterators that yield parsed documents.
\end{itemize}

\subsection{Analyzer}\label{subsec:analyzer}

To process the already parsed documents' text, we have implemented our own Lucene analyzers.
All of them follow the typical workflow: use a \texttt{Tokenizer} and a list of \texttt{TokenFilter} to a
\texttt{TokenStream}.\\

The final version of the project creates an index with four fields for each document.
Because of this, we had to create \textbf{four different analyzers}.
Note that we are not taking into account the first field which is the id, obviously presented in the index, to which no
processing is applied.
All the following described analyzers use some functionalities from the helper class \texttt{AnalyzerUtil} developed by
Nicola Ferro.
We will explain them independently.\\

\subsubsection{English body field}
The processing applied to the English version of the documents (in the class \texttt{EnglishAnalyzer}) is the following:
\begin{enumerate}
    \item Tokenize based on whitespaces.
    \item Eliminate some strange characters found in the documents.
    It is unlikely that a user would perform a query including these characters.
    \item Delete punctuation marks at the beginning and end of words.
    Necessary as we are using the whitespace tokenizer (for example: "address," or "city.").
    \item Apply the \texttt{WordDelimiterGraphFilter} Lucene filter.
    It splits words into subwords and performs optional transformations on subword groups.
    We decided to include the following operations:
    \begin{enumerate}
        \item Divide words into different subwords based on the lower/upper case.
        Example: "PowerShot" converted into tokens "Power" and "Shot".
        \item Divide numbers into different parts based on special characters located in intermediate positions.
        Example: "500-42" converted into "500" and "42".
        \item Concatenate numbers with special characters located in intermediate positions.
        Example "500-42" converted into "50042".
        \item Remove the trailing "s" of English possessives.
        Example: "O'Neil's" converted into "O" and "Neil".
        \item Always maintain the original terms.
        Example: the tokens "PowerShot", "500-42", and "O'Neil's" will be maintained.
    \end{enumerate}
    \item Lowercase all the tokens.
    \item Apply the Terrier~\cite{OunisEtAl2006} stopword list.
    \item Apply query expansion with synonyms using \texttt{SynonymTokenFilter} from Lucene, which is based on the
    WordNet synonym map~\cite{wordnet}.
    A maximum of 10 synonyms can be added to each term.
    \item Apply a minimal stemming process using \texttt{EnglishMinimalStemFilter} from Lucene.
    \item Delete tokens that may have been left empty because of the previous filters.
    For this, we created a custom token filter, \texttt{EmptyTokenFilter} implementing \texttt{FilteringTokenFilter}.
\end{enumerate}

\subsubsection{French body field}
The processing of French documents (in the class \texttt{FrenchAnalyzer}) is identical to the processing of English
documents in the first 5 points (excluding the English possessives' removal in 4.d).
For this point on, we apply:
\begin{enumerate}[start=6]
    \item Apply a French stopword list~\cite{stopword_french}.
    \item Apply a minimal stemming process (in French) using \texttt{FrenchMinimalStemFilter} from Lucene.
    \item Delete empty tokens (\texttt{EmptyTokenFilter}).
\end{enumerate}

\subsubsection{Character N-grams}
Character N-grams are created in the analyzer class \texttt{NGramAnalyzer}, which performs the following operations:
\begin{enumerate}
    \item Tokenize based on whitespaces.
    \item Lowercase all the tokens.
    \item Delete all characters except letters.
    Note that we also maintain the French accent letters.
    After conducting tests on character N-grams with numbers, which resulted in many meaningless numbers being generated, we made the decision to avoid including them.
    \item Delete empty tokens (\texttt{EmptyTokenFilter}).
    \item Generate character N-grams using \texttt{NGramTokenFilter} from Lucene.
\end{enumerate}
The value of N has not been fixed in order to allow for the generation of different experiments.
See Section~\ref{sec:setup} for more details.

\subsubsection{NER extracted information}
The NER information has been extracted using the Apache OpenNLP~\cite{ApacheOpenNLP} library.
As Lucene does not include these functionalities directly, we have used a modified version of a token filter developed
by Nicola Ferro based on the mentioned library, (\texttt{OpenNLPNERFilter}).\\

The processing of the tokens in this analyzer (\texttt{NERAnalyzer}) is the following:
\begin{enumerate}
    \item Tokenize using a standard tokenizer, \texttt{StandardTokenizer} from Lucene.
    \item Apply NER using a tagger model for locations.
    \item Apply NER using a tagger model for person names.
    \item Apply NER using a tagger model for organizations and companies.
\end{enumerate}

\subsection{Index}\label{subsec:index}
%TODO: explain the deprecated indexers
%TODO: review this part
We used the standard Lucene Indexer with the BM25\cite{BM25} similarity.\\
We decide to use NGram analyzer with a multilingual indexer,
in order to index two versions of the same document, creating documents with
unique IDs and bodies from each language.\\
Ngrams are represented as characters from both English and French version of
the documents. N parameter is set at the analyzer level and class is a field,
not stored, in order to minimize space occupation.\\

\subsection{Search}\label{subsec:search}
The search package (only made up of the class \texttt{Searcher}) is in charge of doing the effective search, i.e.,
searching in the created indexes for the topics/queries specified.
In our case we will search the queries provided from LongEval~\cite{traindata}.\\

The searcher will first apply the explained analyzers to the query title, in order to generate a text that can be
matched with the corresponding index fields.
This process is performed by relying on the \texttt{QueryParser} class from Lucene.
After this, the title of the query is searched in the appropriate index fields using the similarity function
BM25~\cite{BM25}.\\

To perform a query, we must specify: the path of the index, the path of the topics file, the number of expected topics,
a run descriptor, and the maximum number of documents to be retrieved (1000).
To facilitate the execution of all experiments, we have created a menu where users can select the desired run to execute
after specifying the required paths and parameters.
This menu determines which document fields and analyzers shall be used, depending on the run identifier.
It also allows distinguishing between train and test data.\\

\subsection{Topic}\label{subsec:topic}

To read the queries (in TREC format) we couldn't use the class \textbf{TrecTopicsReader} already included in Lucene.
This class expects queries to be in a more specific format than the one provided by the organizers.
Thus, we developed our LongEval topic reader in the package \texttt{topic}, containing the classes
\texttt{LongEvalTopic} and \texttt{LongEvalTopicReader}.
Note that we kept the name "TopicReader", but what our \texttt{LongEvalTopicReader} does is reading all the queries,
not the topics.
Thus:
\begin{itemize}
    \item \texttt{LongEvalTopic} is a simple Java POJO representing each query provided by LongEval.
    Each query has a number (\texttt{<num>}) and a title (\texttt{<title>}).
    It is the equivalent of \texttt{QualityQuery} when working with \textbf{TrecTopicsReader}.
    \item \texttt{LongEvalTopicReader} is the query reader we developed.
    It considers the query file as an XML file and parses it using the Java XML library.
\end{itemize}