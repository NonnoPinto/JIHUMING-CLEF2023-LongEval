\section{Methodology}
\label{sec:methodology}
In this section we address how the system was developed starting from the source code of
HelloTipster\cite{tipser} repository developed by Professor Nicola Ferro
and presented to us during the Search Engine course.
Furthermore, we will present main methodology and approaches
using the same structure of the repository\cite{jihuming}.

\subsection{Parsing}
We had a huge collection of documents in English and French. First thing
was to manually examine document to understand how to read them. As described
\href{https://github.com/joaopalotti/trectools}{in his GitHub repository},
documents use a particular format, which includes a DOCNO and a DOCID.
The DOCNO is the id of the document and the DOCID is the id of the collection.\\
JSON, on the other side, follows the mouch more standard
\href{https://github.com/castorini/anserini/issues/1111}{following structure}.\\
The whole parser is made by:
\begin{itemize}
    \item \texttt{DocumentParser}: parses trec document
    \item \texttt{JsonDocument}: create a class for json doc
    \item \texttt{LongEvalParser}: counts the document and print out
    \item \texttt{ParsedDocument}: document parsed has FIELDS including ID, %ENGLISH_BODY, and FRENCH_BODY TODO: for some reason, its an error
\end{itemize}
\subsection{Analyzer}

\subsection{Index}

\subsection{Search}

\subsection{Topic}