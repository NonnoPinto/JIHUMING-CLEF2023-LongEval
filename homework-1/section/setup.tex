\section{Experimental Setup}
\label{sec:setup}

Our work was initiated based on the experimental setups outlined below.
\begin{itemize}
	\item Evaluation measures: MAP (Mean Average Precision) and NDCG (Normalized Discounted Cumulative Gain) scores
	\item Repository: https://bitbucket.org/upd-dei-stud-prj/seupd2223-jihuming.git 
	\item During the development and the experimentation, personal computers were used
	\item Java JDK version 17, Apache version 2, Lucene version 9.5, Maven
\end{itemize}
Our team has created several indexes from the collection which was provided by LongEval. The indexes were created to be used in experiments with the runs, and they were created in order to enable fast and effective search and retrieval. The indexes were created with different implementations of character 3-grams,  character 4-grams and  character 5-grams in order to capture different aspects of the language.The subsequent indexes are:
\begin{itemize}
	\item 2023\_04\_24\_multilingual\_3gram: This index is created with the ability to use, understand, communicate in more than one language since we have two languages in our project. In other words it is proficient in multiple languages. This index is based on groups of three consecutive characters in each document, they are implementing character 3-grams.
	
 \item 2023\_04\_29\_multilingual\_3gram\_synonym: This index is created with the ability to use, understand, communicate in more than one language since we have two languages in our project. In other words it is proficient in multiple languages.  This index is based on groups of three consecutive characters in each document, they are implementing character 3-grams. Additional functionality of this index is having query expansion with synonyms, which provides the improvement of the search results by including related words or phrases that are equivalent in meaning to the user's original query.
	\item 2023\_05\_01\_multilingual\_4gram\_synonym: This index is created with the ability to use, understand, communicate in more than one language since we have two languages in our project. In other words it is proficient in multiple languages. This index uses groups of four consecutive characters in each document to capture additional information about word and phrase structure, they are implementing character 4-grams. Additional functionality of this index is having query expansion with synonyms, which provides the improvement of the search results by including related words or phrases that are equivalent in meaning to the user's original query.
	\item 2023\_05\_01\_multilingual\_5gram\_synonym: This index is created with the ability to use, understand, communicate in more than one language since we have two languages in our project. In other words it is proficient in multiple languages. This index uses groups of five consecutive characters in each document to capture even more detailed information about the structure of words and phrases, they are implementing charter 5-grams.  Additional functionality of this index is having query expansion with synonyms, which provides the improvement of the search results by including related words or phrases that are equivalent in meaning to the user's original query.
	\item 2023\_05\_05\_multilingual\_4gram\_synonym\_ner: This index is created with the ability to use, understand, communicate in more than one language since we have two languages in our project. In other words it is proficient in multiple languages. This index uses groups of four consecutive characters in each document to capture additional information about word and phrase structure, they are implementing character 4-grams. Additional functionality of this index is having query expansion with synonyms, which provides the improvement of the search results by including related words or phrases that are equivalent in meaning to the user's original query. This index uses Named Entity Recognition which provides not only the search for keywords but also identifying and extracting specific named entities. By implementing this to the index, the accuracy and relevance is improved.
\end{itemize}
The indexes also can be found in the following Google Drive folder: 

\href{https://drive.google.com/drive/folders/1CK_kLeZ5Us3VJe8hiG1vhwPrDs94cLvU?usp=share_link}{[9]}