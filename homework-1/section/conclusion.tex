\section{Conclusions and Future Work}
\label{sec:conclusion}

In summary, the IR systems developed in this study followed the Parsing-Analyzer-Index-Search-Topic paradigm and
utilized different methodologies, among which the following stand out: processing of English documents based on
whitespace tokenization, the TERRIER stopword list, query expansion and stemming;
processing of French documents based on whitespace tokenization, a stopword list and stemming;
character N-grams of both versions concatenated;
and NER information extraction using NLP techniques.\\
We evaluated the performance of the 12 systems we developed by measuring the effectiveness of runs on the training data, comprising both French and translated English queries and documents.
To assess the quality of these runs, we used the MAP and NDCG scores calculated by \texttt{trec\_eval}.
Among these systems, five models performed the best, namely \texttt{fr\_fr\_5gram}, \texttt{fr\_fr\_4gram}, \texttt{fr\_fr\_3gram}, \texttt{fr\_fr}, and \texttt{fr\_fr\_4gram\_ner}, listed in order of their scores
from highest to lowest for both MAP and NDCG.\\
In terms of future work, there are several areas that could be explored to improve the effectiveness of the developed IR systems.
Firstly, we could improve indexing methodologies, such as increasing the value of N of N-gram, as we have commented on in Section~\ref{sec:results}.
Secondly, we could explore better NLP techniques to improve the accuracy of the IR systems, as NER turns out not to be very effective. \\
One last possible future work could be a machine-learning based IR system. Using training data, we could train a model to predict the best N for N-grams, the best analyzer, and the best index for a given query and document.
This would be a more dynamic approach to IR systems, as it would be able to adapt to different types of queries and documents.