\section{Experimental Setup}\label{sec:setup}

Our work was initiated based on the experimental setups outlined below.
\begin{itemize}
	\item MAP (Mean Average Precision) and NDCG (Normalized Discounted Cumulative Gain) scores computed for all our
	systems' runs on the training collection, in order to select the five systems to be submitted to CLEF\@.
	\item MAP, NDCG and Rprec (Precision at the Recall
	base) scores computed for the submitted systems' runs on the test (short-term, and long-term) and held-out
	collections, in order to get a reliable estimator of the final performance.
	\item Statistical analysis using Two-Way ANOVA to assess the Average Precision (AP) of the five submitted systems
	across all the topics on the test (short-term, and long-term) and held-out collections.
	\item Pairwise comparisons of the submitted systems using the Tukey Honestly Significant Difference (HSD) test.
	\item All the code and documentation has been developed in our group repository
	(\href{https://bitbucket.org/upd-dei-stud-prj/seupd2223-jihuming/src/master/}{https://bitbucket.org/upd-dei-stud-prj/seupd2223-jihuming/src/master/})~\cite{jihuming}.
	\item During the development and the experimentation, personal computers were used.
	\item Java JDK version 17, Apache version 2, Lucene version 9.5, and Maven.
\end{itemize}

In order to do different run experiments, our team has created several indexes from each of the provided collections
(train, short-term test, and long-term test).
Put simply, certain indexes mentioned in this report incorporate only a few of the characteristics discussed, while
others encompass all the characteristics outlined in the final version of the project.\\

All the created indexes are \textbf{multilingual}, which allows us to take full advantage of the (bilingual) data
collection.
Additionally, we did some experiments with character N-grams generating different versions of indexes with 3-grams,
4-grams and 5-grams;
the motivation was to compare how this parameter affects to the effectiveness of our systems.
3-grams are able to collecting more local information in our documents, while 4-grams and 5-grams are more open to
the context.
An additional functionality of some indexes is query expansion, but as commented, this is only applied to the English
body.
One index includes Named Entity Recognition which provides not only the search for keywords but also identifying and
extracting specific named entities.
The subsequent indexes are:
\begin{itemize}
	\item \texttt{multilingual\_3gram}: both languages of documents, using character 3-grams.
	\item \texttt{multilingual\_3gram\_synonym}: both languages, character 3-grams, (English) query expansion with synonyms.
	\item \texttt{multilingual\_4gram\_synonym}: both languages, character 4-grams, (English) query expansion with synonyms.
	\item \texttt{multilingual\_5gram\_synonym}: both languages, character 5-grams, (English) query expansion with synonyms.
	\item \texttt{multilingual\_4gram\_synonym\_ner}: both languages, character 4-grams, (English) query expansion with synonyms, NER techniques.
\end{itemize}
The indexes also can be found in the following
\href{https://drive.google.com/drive/folders/1CK_kLeZ5Us3VJe8hiG1vhwPrDs94cLvU?usp=share_link}{Google Drive folder}.\\

After creating the indexes, we were able to conduct multiple runs to evaluate the effectiveness of our system.
These runs not only experiment with some techniques specified here, but also consider different versions (English
or French) of the queries.
With them, we can compare and analyze different aspects of our system's performance, such as precision and recall.
The runs are the following:
\begin{itemize}
	\item \texttt{seupd2223-JIHUMING-01\_en\_en}: English topics; using English body field.
	\item \texttt{seupd2223-JIHUMING-02\_en\_en\_3gram}: English topics; using English body field and 3-gram field.
	\item \texttt{seupd2223-JIHUMING-03\_en\_en\_4gram}: English topics; using English body field and 4-gram field.
	\item \texttt{seupd2223-JIHUMING-04\_en\_en\_5gram}: English topics; using English body field and 5-gram field.
	\item \texttt{seupd2223-JIHUMING-05\_en\_en\_fr\_5gram}: English topics; using English and French body fileds and 5-gram field.
	\item \texttt{seupd2223-JIHUMING-06\_en\_en\_4gram\_ner}: English topics; using English body field, 4-gram field and NER information field.
	\item \texttt{seupd2223-JIHUMING-07\_fr\_fr}: French topics; using French body field.
	\item \texttt{seupd2223-JIHUMING-08\_fr\_fr\_3gram}: French topics; using French body field and 3-gram field.
	\item \texttt{seupd2223-JIHUMING-09\_fr\_fr\_4gram}: French topics; using French body field and 4-gram field.
	\item \texttt{seupd2223-JIHUMING-10\_fr\_fr\_5gram}: French topics; using French body field and 5-gram field.
	\item \texttt{seupd2223-JIHUMING-11\_fr\_en\_fr\_5gram}: French topics; using English and French body fields and 5-gram field.
	\item \texttt{seupd2223-JIHUMING-12\_fr\_fr\_4gram\_ner}: French topics; using French body field, 4-gram field and NER information field.
\end{itemize}

The process of creating the indexes typically took around 1 hour, except the indexes that included NER, which took
approximately 16 hours.
On the other hand, generating the runs was a much quicker process, taking consistently less than a minute and a half to
complete.\\

The mentioned analysis of the runs on the training collection will take place in Section~\ref{sec:runs_selection}.
The analysis of the runs on the test collection will take place in Section~\ref{sec:results}.