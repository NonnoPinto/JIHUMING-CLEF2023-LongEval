\section{Methodology}
\label{sec:methodology}

Our final search engine can be divided into the following parts: parsing of the documents and queries, indexing,
text processing (analyzers), and run generation (effective search).\\

Document parsing was performed using the JSON version of the documents.
On the other hand, query parsing was based on an XML parser.\\

In the index, we decided to include four fields: (1) the (processed) English version of the documents, (2) the
(processed) French version, (3) character N-grams of both versions concatenated, and (4) some NER information extracted
from the French (original) version.
As similarity function we have used BM25~\cite{BM25} as it takes into account both term frequency and document length.\\

The text added to the fields must first be processed, for this we have developed four different analyzers.
The English analyzer is based on whitespace tokenization, breaking of words and numbers based on special characters,
lowercasing, applying the Terrier~\cite{OunisEtAl2006} stopword list, query expansion with synonyms based on
the WordNet synonym map~\cite{wordnet}, and stemming.
The French analyzer is based on whitespace tokenization, breaking of words and numbers based on special characters,
lowercasing, applying a French stopword list~\cite{stopword_french} and stemming.
To generate the character N\-grams we consider only the letters of the documents (i.e. we discard numbers and
punctuation).
To perform NER we apply NLP techniques based on Apache OpenNLP~\cite{ApacheOpenNLP}) to the original (French) version of
the documents.
Specifically, we used NER applied to locations, person names and organizations.\\

We conducted some experiments to generate the runs, i.e., we have tried different combinations of the explained
techniques.
Thus, our searcher will always use BM25~\cite{BM25}, but the rest of characteristics depend on the run it is generating.
See Section~\ref{sec:setup} for more details.