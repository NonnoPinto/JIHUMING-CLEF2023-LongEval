\section{Related Work}
\label{sec:related}


There are many search engines, using different techniques to enhance retrieval effectiveness, from which we have taken
inspiration from.
The BM25 similarity function is widely used information retrieval as it considers term frequencies and document length.
This function has demonstrated effectiveness in balancing precision and recall in search results~\cite{BM25}, even if it
doesn't consider meta-data document information as other approaches~\cite{robertson2009probabilistic} do.\\

The tokenization technique we have employed, whitespace tokenization, has not been the primary focus of research in any
study.
Anyway, it seems to provide a useful baseline for tokenization as some studies~\cite{gowsmith2022improving} suggest that
allowing tokens to include spaces produces problem, especially when working with transformers.


In the paper "Learning to Estimate Query Temporal Dynamics for Web Search" ~\cite{cai2014learning}, 
the importance of understanding query temporal 
dynamics for search result ranking is highlighted. By considering the temporal patterns of 
queries, search engines can adapt their ranking algorithms to better meet the evolving information 
needs of users. This can involve giving more weight to recent queries or adjusting the ranking 
based on the popularity of certain topics during specific time periods. By incorporating query 
temporal dynamics into the ranking process, search engines can deliver more relevant and timely 
search results.\\

In the context of our task of assessing an information retrieval system with changing datasets, 
learning and estimating query temporal dynamics can be highly relevant. By leveraging the methods 
and techniques discussed in the paper, it can enhance our information retrieval system's performance 
in addressing evolving user information needs. Understanding the temporal patterns of queries can 
guide the system's adaptation to changes in the datasets, enabling it to provide more accurate and 
timely search results. \\


The paper "Evaluating Web Search Systems Considering Time" ~\cite{hofmann2014evaluating} presents valuable insights into evaluation methodologies for temporal aspects 
in web search systems. While the paper focuses specifically on web search, many of the concepts and 
methodologies discussed can be applied to the task of assessing an information retrieval system with 
changing datasets. \\

Specifically, the paper explores metrics for evaluating retrieval effectiveness over time. These metrics 
can be utilized to measure the performance of your information retrieval system in handling the changing 
datasets. Consider incorporating relevant metrics, such as precision, recall, F1-score, and mean average 
precision (MAP), as discussed in the paper, to evaluate the system's effectiveness in retrieving relevant 
information across different time periods.\\

For experimental setups, the paper provides insights into various setups, including time-sliced evaluation, 
incremental evaluation, and evaluation with simulated temporal queries. These setups can serve as a basis 
for designing your own experimental setups that align with your specific task requirements.\\

By leveraging the insights from the paper, we can incorporate evaluation methodologies, metrics, and
experimental setups specifically tailored for temporal information retrieval. 
This comprehensive approach will enable you to assess the performance of your information retrieval 
system in handling changes over time, evaluating its adaptability to evolving datasets and user queries.\\
