\section{Related Work}
\label{sec:related}

There are many search engines using different techniques to enhance retrieval effectiveness from which we have taken
inspiration from.\\

% About our techniques
The BM25~\cite{BM25} similarity function is widely used information retrieval as it considers term frequencies and
document length.
This function has demonstrated effectiveness in balancing precision and recall in search results, even if it
doesn't consider meta-data document information as other approaches~\cite{robertson2009probabilistic} do.
Whitespace tokenization has not been the primary focus of research in any study, anyway, it seems to provide a useful
baseline for tokenization.
E. Gow-Smith et al.~\cite{gowsmith2022improving} suggests that allowing tokens to include spaces causes problems,
especially in architectures including transformers.
Token lowercasing is also a recurring method in information retrieval~\cite{manning2008introduction}, mainly because it
reduces the vocabulary size.\\

The Terrier~\cite{OunisEtAl2006} stopword list has been used in plenty of search engines because of the good results if
offers working with web documents as blogs~\cite{ounis2009overview} or even recommender systems.
Another basic information retrieval technique used in our search engine has been stemming.
We have relied on the work of A. G. Jivani et al.\cite{jivani2011comparative} to get an overview of the most adequate
stemming techniques for our documents.
For the English documents we have chosen a minimal stemmer developed by D. K. Harman~\cite{Harman1991HowEI}.
For the French documents we have also used a minimal stemmer developed by J. Savoy~\cite{frenchStemmer}.\\

We have used query expansion~\cite{efthimiadis1996query} in order to broaden the search scope by including synonyms
related to the original query.
These synonyms come from WordNet~\cite{Fellbaum1998}, a popular lexical database that provides semantic relationships
between words.
We have also included character N-grams of the English and French versions of the documents.
Our experiments on character N-grams have been focus on comparing how the value of \textit{N} can affect to the
retrieval effectiveness.
Our motivation for this study stemmed from the works of T. Wilson et al.~\cite{wilson2008comparing}, and J.
Goodman~\cite{goodman2001bit}, which also explored the impact of different N-gram models (among others) on performance.
We tried to refine our results including Named Entity Recognition~\cite{mohit2014named}.
This technique has proven useful in other information retrieval systems addressing for example the
food~\cite{popovski2020survey} or the archaeology~\cite{brandsen2022can} domain.\\

% About our approach
The work from F. Cai et al.~\cite{cai2014learning} remarks the importance of understanding query temporal dynamics for
search result ranking.
By considering the temporal patterns of queries and incorporating query temporal dynamics into the ranking process,
search engines can deliver more relevant and timely results.
An example of this is giving more weight to recent queries or adjusting the ranking based on a certain popularity during
specific time periods.
In the context of our task, with changing datasets, learning and estimating query temporal dynamics can be highly
relevant.\\

The work from K. Hofmann et al.~\cite{hofmann2014evaluating} presents valuable insights into evaluation methodologies for
temporal aspects in web search systems.
Specifically, the paper explores metrics for evaluating retrieval effectiveness over time, such as precision, recall,
F-1 score, and mean average precision (MAP).
The paper provides insights into various setups, including time-sliced evaluation, incremental evaluation, and
evaluation with simulated temporal queries.
These setups can serve as a basis for designing experimental setups that align with specific task requirements.
With this motivation we can incorporate evaluation methodologies, metrics, and experimental setups specifically tailored
for temporal information retrieval.\\
