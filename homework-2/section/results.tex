\newpage
\section{Results and Discussion}
\label{sec:results}

\subsection{Short Term Test Data}\label{subsec:short_term}

In Table~\ref{tab:st_scores} we can see the mentioned scores computed for our five submitted systems on the (test)
short-term collection.\\

Based on the Two-Way ANOVA analysis presented in Table~\ref{tab:st_anova}, we observed a significant p-value (p < 0.05).
Thus, we can conclude that there are significant differences among our systems.
As ANOVA does not tell which systems are significantly different from each other, in Table~\ref{tab:st_comparison} we
can observe the Tukey’s honestly significantly differenced (HSD) test.
It suggests that pairwise comparisons between systems 7--12, 8--12, 9--12 and 10--12 reject null hypothesis (p < 0.05)
and indicates statistical significant differences.

\begin{table}[h!]
    \begin{center}
        \caption{MAP, NCDG and Rprec scores for submitted runs in test data, short-term evaluation}
        \label{tab:st_scores}
        \begin{tabular}{|c|c||c|c|c|}
            \hline
            \textbf{Run ID} & \textbf{Run} & \textbf{NCDG Score} & \textbf{MAP Score} & \textbf{RPREC Score}\\
            \hline\hline
            07 & fr\_fr & 0.3367 & 0.1883 & 0.1561 \\
            \hline
            08 & fr\_fr\_3gram & 0.3384 & 0.1893 & 0.1579 \\
            \hline
            09 & fr\_fr\_4gram & 0.3423 & 0.1911 & 0.1581 \\
            \hline
            10 & fr\_fr\_5gram & 0.3447 & 0.1926 & 0.1603 \\
            \hline
            12 & fr\_fr\_4gram\_ner & 0.2980 & 0.1468 & 0.1172 \\
            \hline
        \end{tabular}
    \end{center}
\end{table}

\begin{table}[h!]
    \centering
    \caption{Two-Way ANOVA table assessing AP for the submitted systems on the (test) short-term collection}
    \label{tab:st_anova}
    \begin{tabular}{|l|l|l|l|l|l|}
    \hline
        \textbf{Source} & \textbf{df} & \textbf{sum\_sq} & \textbf{mean\_sq} & \textbf{F} & \textbf{PR(>F)} \\ \hline\hline
        \textbf{C(system)} & 4 & 1,348254 & 0,337063 & 6,250613 & 0,000052 \\ \hline
        \textbf{Residual} & 4405 & 237,539013 & 0,053925 & -- & -- \\ \hline
    \end{tabular}
\end{table}

\begin{table}[h!]
    \centering
    \caption{Tukey Honestly Significant Difference test for the submitted systems on the (test) short-term collection}
    \label{tab:st_comparison}
    \begin{tabular}{|l|l||c|c|c|c|c|}
        \hline
        \textbf{Run 1} & \textbf{Run 2} & \textbf{Diff} & \textbf{Lower} & \textbf{Upper} & \textbf{q-value} & \textbf{p-value} \\ \hline
        07\_fr\_fr & 08\_fr\_fr\_3gram & 0,000986 & -0,029190 & 0,031162 & 0,126122 & 0,900000 \\ \hline
        07\_fr\_fr & 09\_fr\_fr\_4gram & 0,002808 & -0,027368 & 0,032984 & 0,359124 & 0,900000 \\ \hline
        07\_fr\_fr & 10\_fr\_fr\_5gram & 0,004282 & -0,025894 & 0,034458 & 0,547611 & 0,900000 \\ \hline
        07\_fr\_fr & 12\_fr\_fr\_4gram\_ner & 0,041538 & 0,011362 & 0,071714 & 5,312288 & 0,001641 \\ \hline
        08\_fr\_fr\_3gram & 09\_fr\_fr\_4gram & 0,001822 & -0,028354 & 0,031998 & 0,233002 & 0,900000 \\ \hline
        08\_fr\_fr\_3gram & 10\_fr\_fr\_5gram & 0,003296 & -0,026880 & 0,033472 & 0,421489 & 0,900000 \\ \hline
        08\_fr\_fr\_3gram & 12\_fr\_fr\_4gram\_ner & 0,042524 & 0,012348 & 0,072700 & 5,438410 & 0,001152 \\ \hline
        09\_fr\_fr\_4gram & 10\_fr\_fr\_5gram & 0,001474 & -0,028702 & 0,031650 & 0,188487 & 0,900000 \\ \hline
        09\_fr\_fr\_4gram & 12\_fr\_fr\_4gram\_ner & 0,044346 & 0,014170 & 0,074522 & 5,671412 & 0,001000 \\ \hline
        10\_fr\_fr\_5gram & 12\_fr\_fr\_4gram\_ner & 0,045820 & 0,015644 & 0,075995 & 5,859899 & 0,001000 \\ \hline
    \end{tabular}
\end{table}

\subsection{Long Term Test Data}\label{subsec:long_term}

In Table~\ref{tab:lt_scores} we can see the mentioned scores computed for our five submitted systems on the (test)
long-term collection.\\

In the Two-Way ANOVA analysis presented in Table~\ref{tab:lt_anova} we observed a significant p-value (p < 0.05), so we
can conclude that there are significant differences among our systems.
The Tukey's honestly significantly differenced (HSD) test in Table~\ref{tab:lt_comparison} again suggests that pairwise
comparisons between systems 7--12, 8--12, 9--12 and 10--12 reject null hypothesis (p < 0.05).
Thus system 12 presents statistical significant difference.

\begin{table}[h!]
    \begin{center}
        \caption{MAP, NCDG and Rprec scores for submitted runs in test data, long-term evaluation}
        \label{tab:lt_scores}
        \begin{tabular}{|c|c||c|c|c|}
            \hline
            \textbf{Run ID} & \textbf{Run} & \textbf{NCDG Score} & \textbf{MAP Score} & \textbf{RPREC Score}\\
            \hline\hline
            07 & fr\_fr & 0.3447 & 0.1880 & 0.1589 \\
            \hline
            08 & fr\_fr\_3gram & 0.3454 & 0.1881 & 0.1600 \\
            \hline
            09 & fr\_fr\_4gram & 0.3480 & 0.1888 & 0.1611 \\
            \hline
            10 & fr\_fr\_5gram & 0.3533 & 0.1920 & 0.1642 \\
            \hline
            12 & fr\_fr\_4gram\_ner & 0.3046 & 0.1433 & 0.1192\\
            \hline
        \end{tabular}
    \end{center}
\end{table}

\begin{table}[!ht]
    \centering
    \caption{Two-Way ANOVA table assessing AP for the submitted systems on the (test) long-term collection}
    \label{tab:lt_anova}
    \begin{tabular}{|l|l|l|l|l|l|}
    \hline
        \textbf{Source} & \textbf{df} & \textbf{sum\_sq} & \textbf{mean\_sq} & \textbf{F} & \textbf{PR(>F)} \\ \hline\hline
        \textbf{C(system)} & 4 & 1,564386 & 0,391097 & 8,562506 & 7,01E-07 \\ \hline
        \textbf{Residual} & 4610 & 210,56392 & 0,045675 & NaN & NaN \\ \hline
    \end{tabular}
\end{table}

\begin{table}[h!]
    \centering
    \caption{Tukey Honestly Significant Difference test for the submitted systems on the (test) long-term collection}
    \label{tab:lt_comparison}
    \begin{tabular}{|l|l||c|c|c|c|c|}
        \hline
        \textbf{Run 1} & \textbf{Run 2} & \textbf{Diff} & \textbf{Lower} & \textbf{Upper} & \textbf{q-value} & \textbf{p-value} \\ \hline
        07\_fr\_fr & 08\_fr\_fr\_3gram & 0,000340 & -0,026808 & 0,027488 & 0,048345 & 0,900000 \\ \hline
        07\_fr\_fr & 09\_fr\_fr\_4gram & 0,001090 & -0,026058 & 0,028237 & 0,154891 & 0,900000 \\ \hline
        07\_fr\_fr & 10\_fr\_fr\_5gram & 0,004239 & -0,022909 & 0,031387 & 0,602592 & 0,900000 \\ \hline
        07\_fr\_fr & 12\_fr\_fr\_4gram\_ner & 0,044458 & 0,017311 & 0,071606 & 6,319943 & 0,001000 \\ \hline
        08\_fr\_fr\_3gram & 09\_fr\_fr\_4gram & 0,000750 & -0,026398 & 0,027897 & 0,106546 & 0,900000 \\ \hline
        08\_fr\_fr\_3gram & 10\_fr\_fr\_5gram & 0,003899 & -0,023249 & 0,031047 & 0,554247 & 0,900000 \\ \hline
        08\_fr\_fr\_3gram & 12\_fr\_fr\_4gram\_ner & 0,044798 & 0,017651 & 0,071946 & 6,368287 & 0,001000 \\ \hline
        09\_fr\_fr\_4gram & 10\_fr\_fr\_5gram & 0,003149 & -0,023998 & 0,030297 & 0,447701 & 0,900000 \\ \hline
        09\_fr\_fr\_4gram & 12\_fr\_fr\_4gram\_ner & 0,045548 & 0,018400 & 0,072696 & 6,474833 & 0,001000 \\ \hline
        10\_fr\_fr\_5gram & 12\_fr\_fr\_4gram\_ner & 0,048697 & 0,021550 & 0,075845 & 6,922534 & 0,001000 \\ \hline
    \end{tabular}
\end{table}

\subsection{Held Out Test Data}\label{subsec:held_out}

In Table~\ref{tab:ho_scores} we can see the mentioned scores computed for our five submitted systems on the (train)
held-out collection.\\

In the Two-Way ANOVA analysis presented in Table~\ref{tab:ho_anova} we didn't observe a significant p-value (p >= 0.05),
so we can conclude that there are not significant differences among our systems.

\begin{table}[h!]
    \begin{center}
        \caption{MAP, NCDG and Rprec scores for submitted runs in training data, held-out evaluation}
        \label{tab:ho_scores}
        \begin{tabular}{|c|c||c|c|c|}
            \hline
            \textbf{Run ID} & \textbf{Run} & \textbf{NCDG Score} & \textbf{MAP Score} & \textbf{RPREC Score}\\
            \hline\hline
            07 & fr\_fr & 0.3271 & 0.1746 & 0.1397 \\
            \hline
            08 & fr\_fr\_3gram & 0.3307 & 0.1725 & 0.1326 \\
            \hline
            09 & fr\_fr\_4gram & 0.3364 & 0.1763 & 0.1397 \\
            \hline
            10 & fr\_fr\_5gram & 0.3413 & 0.1788 & 0.1385 \\
            \hline
            12 & fr\_fr\_4gram\_ner & 0.2868 & 0.1369 & 0.1004 \\
            \hline
        \end{tabular}
    \end{center}
\end{table}

\begin{table}[h!]
    \centering
    \caption{Two-Way ANOVA table assessing AP on the held-out collection}
    \label{tab:ho_anova}
    \begin{tabular}{|l|l|l|l|l|l|}
    \hline
        \textbf{Source} & \textbf{df} & \textbf{sum\_sq} & \textbf{mean\_sq} & \textbf{F} & \textbf{PR(>F)} \\ \hline\hline
        \textbf{C(system)} & 4 & 0,11896 & 0,02974 & 0,689161 & 0,599712 \\ \hline
        \textbf{Residual} & 485 & 20,929668 & 0,043154 & -- & -- \\ \hline
    \end{tabular}
\end{table}

%\begin{table}[h!]
%    \centering
%    \caption{Other scores}
%    \label{tab:ho_comparison}
%    \begin{tabular}{|r|l|l||c|c|c|c|c|}
%    \hline
%        \multicolumn{8}{|l|}{\textbf{HELD-OUT SECTION}} \\ \hline\hline
%        \textbf{ID} & \textbf{group 1} & \textbf{group 2} & \textbf{Diff} & \textbf{Lower} & \textbf{Upper} & \textbf{q-value} & \textbf{p-value} \\ \hline
%        0 & 07\_fr\_fr & 08\_fr\_fr\_3gram & 0,002054 & -0,079203 & 0,083311 & 0,097886 & 0,900000 \\ \hline
%        1 & 07\_fr\_fr & 09\_fr\_fr\_4gram & 0,001704 & -0,079553 & 0,082961 & 0,081207 & 0,900000 \\ \hline
%        2 & 07\_fr\_fr & 10\_fr\_fr\_5gram & 0,004244 & -0,077013 & 0,085501 & 0,202239 & 0,900000 \\ \hline
%        3 & 07\_fr\_fr & 12\_fr\_fr\_4gram\_ner & 0,037636 & -0,043621 & 0,118892 & 1,793506 & 0,685812 \\ \hline
%        4 & 08\_fr\_fr\_3gram & 09\_fr\_fr\_4gram & 0,003758 & -0,077499 & 0,085015 & 0,179093 & 0,900000 \\ \hline
%        5 & 08\_fr\_fr\_3gram & 10\_fr\_fr\_5gram & 0,006298 & -0,074959 & 0,087555 & 0,300125 & 0,900000 \\ \hline
%        6 & 08\_fr\_fr\_3gram & 12\_fr\_fr\_4gram\_ner & 0,035582 & -0,045675 & 0,116838 & 1,695620 & 0,725010 \\ \hline
%        7 & 09\_fr\_fr\_4gram & 10\_fr\_fr\_5gram & 0,002540 & -0,078717 & 0,083797 & 0,121032 & 0,900000 \\ \hline
%        8 & 09\_fr\_fr\_4gram & 12\_fr\_fr\_4gram\_ner & 0,039340 & -0,041917 & 0,120597 & 1,874713 & 0,653298 \\ \hline
%        9 & 10\_fr\_fr\_5gram & 12\_fr\_fr\_4gram\_ner & 0,041880 & -0,039377 & 0,123136 & 1,995746 & 0,604835 \\ \hline
%    \end{tabular}
%\end{table}

\subsection{General results}\label{subsec:general-results}

Results suggest that French queries perform better than their English counterparts, possibly due to the training data's
French origin and later translation into English.
Moreover, the IR system's effectiveness generally increases with a larger N-gram size, as indicated by the higher scores
of \texttt{en\_en\_5gram} and \texttt{fr\_fr\_5gram}.
Conversely, the inclusion of NER in the indexing process seems to have a negative impact on the scores, as shown by the
lower scores of \texttt{en\_en\_4gram\_ner} and \texttt{fr\_fr\_4gram\_ner}.
The use of query expansion with synonyms in English does not seem to improve the search results to any great extent.\\

It's interesting to notice that the cross-language approaches (\texttt{en\_en\_fr\_5gram} and
\texttt{fr\_en\_fr\_5gram}) are out of the five bests systems.
It turns out that searching for English words in French documents and vice versa messes up the search, lowering the
score.
Another interesting aspect is that the worst-performing index is the one with named entity recognition in English
(\texttt{en\_en\_4gram\_ner}): it combines translated queries and NER, which appears to be the two worst-performing
approaches.\\

In general, we focus more on trying multiple approaches, this is why our score has such a big space for improvement.
As already said, French queries with bigger N-gram sizes perform better.
Instead of relying on single-word matches, the queries could take place with more context, resulting in better search
results.\\

\newpage